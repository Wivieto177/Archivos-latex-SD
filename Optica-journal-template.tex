\documentclass[9pt,twocolumn,twoside]{opticajnl}
\journal{opticajournal}

\setboolean{shortarticle}{false}

\title{Resumen del artículo 'Computación en la nube para sistemas distribuidos'}

\author[1]{Bolívar Ramos Mosquera}
\affil[*]{Resumen hecho porFernanda Elizabeth Basurto Muñoz y Willian Javier Cedeño Bravo}

\affil[1]{Universidad de Guayaquil, Ecuador.}


\doi{}

\begin{abstract}
Este trabajo presenta un resumen detallado del artículo titulado 'Computación en la nube para sistemas distribuidos'. La investigación propone una estrategia educativa orientada al uso de **PaaS (Plataforma como Servicio)**, con el objetivo de que los estudiantes de Ingeniería en Sistemas Computacionales desarrollen competencias prácticas en diseño arquitectónico e implementación de sistemas distribuidos. El uso de la nube pública como entorno de desarrollo permite a los estudiantes ejecutar proyectos ligeros de software distribuido, mejorando la comprensión de conceptos clave. La estrategia incluye seis fases: diagnóstico, planificación, capacitación, implementación, evaluación y reutilización. Este enfoque facilita la aplicación práctica de los conocimientos teóricos, abordando problemas identificados en la enseñanza de sistemas distribuidos.
\end{abstract}

\setboolean{displaycopyright}{false}

\begin{document}

\maketitle

\section{Introducción}
El desarrollo de sistemas distribuidos es una asignatura compleja en los programas de **Ingeniería en Sistemas Computacionales**, pues implica conceptos teóricos difíciles de asimilar. Los estudiantes enfrentan retos para comprender estos sistemas debido a su naturaleza abstracta y difusa. El autor señala la importancia de aplicar un enfoque práctico que permita a los estudiantes implementar sistemas distribuidos ligeros para reforzar los conceptos teóricos.

La **Computación en la Nube**, específicamente **PaaS (Plataforma como Servicio)**, se presenta como una solución efectiva para estos desafíos educativos. PaaS proporciona un entorno flexible que permite a los estudiantes diseñar, implementar y experimentar con sistemas distribuidos sin las complejidades de gestionar infraestructuras físicas. Plataformas como **IBM Cloud Lite**, **Google App Engine**, **AWS Elastic Beanstalk**, y **Azure App Service** son ejemplos recomendados en el artículo.

\section{Materiales y Métodos}
La estrategia educativa propuesta en el artículo se divide en seis fases clave, orientadas a proporcionar un aprendizaje progresivo y práctico:

1. **Diagnóstico**: En esta fase inicial, los estudiantes son evaluados en áreas como arquitectura de software, patrones arquitectónicos y metodologías ágiles. A través de ejercicios de modelado utilizando **UML (Unified Modelling Language)**, se identifican sus fortalezas y áreas de mejora, lo que permite planificar los siguientes pasos de manera personalizada.

2. **Planificación**: Basándose en el diagnóstico, se establecen objetivos de aprendizaje específicos y se planifican las iteraciones de los proyectos. Cada iteración tiene una duración de tres meses e incluye la enseñanza teórica y práctica en laboratorios. Los proyectos involucran la utilización de servicios en la nube, permitiendo a los estudiantes aplicar los conceptos teóricos en entornos controlados.

3. **Capacitación**: Los estudiantes reciben formación teórica en diseño arquitectónico y **sistemas distribuidos**, así como sesiones prácticas utilizando **IBM Cloud Lite**. La capacitación incluye el uso de servicios en la nube como **Cloud Functions** y **Liberty for Java**, permitiendo a los estudiantes adquirir competencias técnicas en un entorno seguro y controlado.

4. **Implementación**: En esta fase, los estudiantes desarrollan proyectos utilizando metodologías ágiles como **DevOps**. El uso de plataformas PaaS facilita la implementación de los sistemas distribuidos, alojando el backend en la nube y manteniendo el cliente localmente. Tecnologías como **Node.js** y **Db2** son empleadas para la gestión de bases de datos y la ejecución de servicios backend.

5. **Evaluación**: La evaluación es continua, centrándose tanto en la calidad técnica de los proyectos como en la capacidad de los estudiantes para aplicar los conceptos teóricos. Los docentes supervisan en línea los proyectos implementados en la nube, revisando métricas como tiempo de ejecución y funcionalidad.

6. **Reutilización**: Los proyectos exitosos se reutilizan como material educativo para futuras iteraciones, permitiendo que los estudiantes de periodos posteriores aprendan de los ejemplos desarrollados. Esto incluye el uso de los proyectos para enseñar sobre escalabilidad, rendimiento y calidad del software.

\section{Resultados y Discusión}
La implementación de esta estrategia educativa mostró resultados positivos, pero también desafíos. El 75\% de los estudiantes demostró buen dominio en la gestión de requisitos, mientras que el 87\% presentó dificultades en el diseño arquitectónico inicial. A lo largo de las iteraciones, las sesiones adicionales de capacitación ayudaron a mejorar estas deficiencias.

No obstante, algunos estudiantes enfrentaron dificultades con el uso de **IBM Cloud Lite**. Durante la primera iteración, el 55\% de los estudiantes encontró la plataforma "muy difícil" de usar debido a la falta de familiaridad con las herramientas en la nube. Sin embargo, tras ajustar la estrategia y aumentar la cantidad de laboratorios prácticos, el 65\% de los estudiantes consideró que el uso de la plataforma fue "fácil" en la segunda iteración.

En términos de proyectos, los estudiantes lograron implementar sistemas distribuidos ligeros en la nube, mejorando su comprensión en aspectos de escalabilidad y comunicación cliente-servidor. A medida que los estudiantes se familiarizaron con las herramientas, lograron una implementación más eficiente en la segunda iteración.

\section{Evaluación de la Propuesta}
La evaluación de la estrategia se realizó mediante encuestas basadas en el **Modelo de Aceptación de Tecnología (TAM)**, donde se identificó que la mayoría de los estudiantes percibió la utilidad de implementar proyectos ligeros en la nube para mejorar su comprensión de los conceptos de sistemas distribuidos y diseño arquitectónico. A pesar de las dificultades iniciales, los estudiantes mostraron un incremento significativo en su confianza para trabajar en entornos en la nube, y se observó una mejora en la calidad de los proyectos desarrollados.

Además, los docentes notaron que el uso de plataformas PaaS públicas facilitó la evaluación de los proyectos, ya que permitía verificar la ejecución de los sistemas en tiempo real sin necesidad de compilaciones locales.

\section{Conclusiones}
La adopción de plataformas **PaaS** como **IBM Cloud Lite** en la enseñanza de sistemas distribuidos ha demostrado ser una herramienta efectiva para superar las barreras teóricas y prácticas en el aprendizaje de los estudiantes. Aunque los estudiantes enfrentaron desafíos al principio, los resultados finales muestran que este enfoque fortalece sus habilidades teóricas y prácticas, preparándolos mejor para proyectos más complejos. Se recomienda continuar utilizando este enfoque en la educación en sistemas distribuidos y expandir el uso de servicios en la nube para fomentar un aprendizaje práctico.

\begin{backmatter}
\bmsection{Funding}
Este trabajo no recibió financiamiento externo.

\bmsection{Acknowledgment}
Resumen basado en el artículo de Bolívar Ramos Mosquera.

\bmsection{Disclosures}
Los autores no presentan conflictos de interés.
\end{backmatter}

\end{document}
